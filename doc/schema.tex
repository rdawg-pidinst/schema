\documentclass[titlepage=true,twoside=true,DIV=11]{scrartcl}
\usepackage[utf8]{inputenc}
\usepackage[T1]{fontenc}
\usepackage[colorlinks=true,urlcolor=black,linkcolor=black]{hyperref}

\input{.meta}

\title{A Metadata Schema for the \\
  Persistent Identification of Instruments}
\author{RDA Persistent Identification of Instruments WG}
\date{\schemadate}

\begin{document}

\maketitle

\begin{center}
  \begin{tabular}[t]{|l|l|}
    \hline
    Version & \schemaversion \\
    \hline
    Date    & \schemadate    \\
    \hline
  \end{tabular}
\end{center}

\cleardoublepage
\pagenumbering{roman}

\section*{Contributors}

\begin{itemize}\emergencystretch 3em
\item \emph{Rolf Krahl}
  ({\small
  \href{mailto:rolf.krahl@helmholtz-berlin.de}{rolf.krahl@helmholtz-berlin.de},
  \url{https://orcid.org/0000-0002-1266-3819}}),
  Helmholtz-Zentrum Berlin für Materialien und Energie,
  Albert-Einstein-Str.\ 15, 12489 Berlin, Germany
\item \emph{Louise Darroch}
  ({\small
  \href{mailto:louise.darroch@bodc.ac.uk}{louise.darroch@bodc.ac.uk},
  \url{https://orcid.org/0000-0003-4163-9575}}),
  British Oceanographic Data Centre, National Oceanography Centre,
  Liverpool, L3 5DA, United Kingdom
\item \emph{Robert Huber}
  ({\small
  \href{mailto:rhuber@uni-bremen.de}{rhuber@uni-bremen.de},
  \url{https://orcid.org/0000-0003-3000-0020}}),
  MARUM - Center for Marine Environmental Sciences, University of Bremen,
  Leobener Str.\ 8, 28359 Bremen, Germany
\item \emph{Anusuriya Devaraju}
  ({\small
  \href{mailto:adevaraju@marum.de}{adevaraju@marum.de},
  \url{https://orcid.org/0000-0003-0870-3192}}),
  MARUM - Center for Marine Environmental Sciences, University of Bremen,
  Leobener Str.\ 8, 28359 Bremen, Germany
\item \emph{Ted Habermann}
  ({\small
  \href{mailto:ted@tedhabermann.com}{ted@tedhabermann.com},
  \url{http://orcid.org/0000-0003-3585-6733}}),
  Metadata Game Changers,
  3980 Broadway, Suite 103-185, Boulder, Colorado, USA 80304
\item \emph{Markus Stocker}
  ({\small
  \href{mailto:markus.stocker@tib.eu}{markus.stocker@tib.eu},
  \url{https://orcid.org/0000-0001-5492-3212}}),
  TIB Leibniz Information Centre for Science and Technology,
  Welfengarten 1 B, 30167 Hannover, Germany and
  PANGAEA, Center for Marine Environmental Sciences (MARUM),
  University of Bremen, Leobener Str.\ 8, 28359 Bremen, Germany
\item \emph{The Research Data Alliance Persistent Identification of Instruments
  Working Group members}
  ({\small \url{https://www.rd-alliance.org/node/57186/members}})
\end{itemize}

\tableofcontents
\cleardoublepage
\pagenumbering{arabic}

\section{Introduction}

The Persistent Identification of Instruments WG (PIDINST) seeks to
explore a community-driven solution for globally unique identification
of measuring instruments operated in the sciences.

Measuring instruments, such as sensors used in environmental science,
DNA sequencers used in life sciences or laboratory engines used for
medical domains, are widespread in most fields of applied sciences.
The ability to link an active instrument (instance) with an instrument
type and with the broader context in which the instrument operates,
including generated data, other instruments and platforms, people and
manufacturers, etc., is critical, especially for automated processing
of such contextual information and for the interpretation of generated
data.

PIDINST is a working group in the Research Data Alliance (RDA).  It
aims to establish a cross-discipline, operational solution for the
unique and lasting identification of measuring instruments actively
operated in the sciences.  See \cite{pidinst2020} for an overview of
the work of PIDINST.

The present document defines the metadata schema that PIDINST
developed for the description of an instrument instance.  These
metadata items are supposed to be registered along with the persistent
identifier and stored in the PID infrastructure.

\section{Metadata Schema}

\ldots

\bibliographystyle{plain}
\bibliography{schema}

\end{document}
