\documentclass[titlepage=true,twoside=false,DIV=13]{scrartcl}
\usepackage[utf8]{inputenc}
\usepackage[T1]{fontenc}
\usepackage[colorlinks=true,urlcolor=black,linkcolor=black]{hyperref}
\usepackage{longtable}
\usepackage{enumitem}

\input{.meta}

\title{A Metadata Schema for the \\
  Persistent Identification of Instruments}
\author{RDA Persistent Identification of Instruments WG}
\date{\schemadate}

\begin{document}

\maketitle

\begin{center}
  \begin{tabular}[t]{|l|l|}
    \hline
    Version & \schemaversion \\
    \hline
    Date    & \schemadate    \\
    \hline
  \end{tabular}
\end{center}

\cleardoublepage
\pagenumbering{roman}

\section*{Contributors}

\begin{itemize}\emergencystretch 3em
\item \emph{Rolf Krahl}
  ({\small
  \href{mailto:rolf.krahl@helmholtz-berlin.de}{rolf.krahl@helmholtz-berlin.de},
  \url{https://orcid.org/0000-0002-1266-3819}}),
  Helmholtz-Zentrum Berlin für Materialien und Energie,
  Albert-Einstein-Str.\ 15, 12489 Berlin, Germany
\item \emph{Louise Darroch}
  ({\small
  \href{mailto:louise.darroch@bodc.ac.uk}{louise.darroch@bodc.ac.uk},
  \url{https://orcid.org/0000-0003-4163-9575}}),
  British Oceanographic Data Centre, National Oceanography Centre,
  Liverpool, L3 5DA, United Kingdom
\item \emph{Robert Huber}
  ({\small
  \href{mailto:rhuber@uni-bremen.de}{rhuber@uni-bremen.de},
  \url{https://orcid.org/0000-0003-3000-0020}}),
  MARUM - Center for Marine Environmental Sciences, University of Bremen,
  Leobener Str.\ 8, 28359 Bremen, Germany
\item \emph{Anusuriya Devaraju}
  ({\small
  \href{mailto:adevaraju@marum.de}{adevaraju@marum.de},
  \url{https://orcid.org/0000-0003-0870-3192}}),
  MARUM - Center for Marine Environmental Sciences, University of Bremen,
  Leobener Str.\ 8, 28359 Bremen, Germany
\item \emph{Ted Habermann}
  ({\small
  \href{mailto:ted@tedhabermann.com}{ted@tedhabermann.com},
  \url{http://orcid.org/0000-0003-3585-6733}}),
  Metadata Game Changers,
  3980 Broadway, Suite 103-185, Boulder, Colorado, USA 80304
\item \emph{Markus Stocker}
  ({\small
  \href{mailto:markus.stocker@tib.eu}{markus.stocker@tib.eu},
  \url{https://orcid.org/0000-0001-5492-3212}}),
  TIB Leibniz Information Centre for Science and Technology,
  Welfengarten 1 B, 30167 Hannover, Germany and
  PANGAEA, Center for Marine Environmental Sciences (MARUM),
  University of Bremen, Leobener Str.\ 8, 28359 Bremen, Germany
\item \emph{The Research Data Alliance Persistent Identification of Instruments
  Working Group members}
  ({\small \url{https://www.rd-alliance.org/node/57186/members}})
\end{itemize}

\tableofcontents
\cleardoublepage
\pagenumbering{arabic}

\section{Introduction}

The Persistent Identification of Instruments WG (PIDINST) seeks to
explore a community-driven solution for globally unique identification
of measuring instruments operated in the sciences.

Measuring instruments, such as sensors used in environmental science,
DNA sequencers used in life sciences or laboratory engines used for
medical domains, are widespread in most fields of applied sciences.
The ability to link an active instrument (instance) with an instrument
type and with the broader context in which the instrument operates,
including generated data, other instruments and platforms, people and
manufacturers, etc., is critical, especially for automated processing
of such contextual information and for the interpretation of generated
data.

PIDINST is a working group in the Research Data Alliance (RDA).  It
aims to establish a cross-discipline, operational solution for the
unique and lasting identification of measuring instruments actively
operated in the sciences.  See \cite{pidinst2020} for an overview of
the work of PIDINST.

The present document defines the metadata schema that PIDINST
developed for the description of an instrument instance.  These
metadata items are supposed to be registered along with the persistent
identifier and stored in the PID infrastructure.

\section{Metadata Schema}

\newlength{\idcolw}\settowidth{\idcolw}{5.3.1}
\newlength{\propcolw}\settowidth{\propcolw}{instrumentTypeIdentifierType}
\newlength{\occcolw}\settowidth{\occcolw}{Occ}
\newlength{\valcolw}\settowidth{\valcolw}{Controlled list of values:}
\newlength{\defcolw}
\setlength{\defcolw}{\textwidth}
\addtolength{\defcolw}{-\idcolw}
\addtolength{\defcolw}{-\propcolw}
\addtolength{\defcolw}{-\occcolw}
\addtolength{\defcolw}{-\valcolw}
\addtolength{\defcolw}{-10\tabcolsep}

\begin{longtable}{|l|l|l|p{\defcolw}|p{\valcolw}|}
  \hline
  ID    & Property                     & Occ
        & Definition & Allowed values, constraints, remarks \\
  \hline \endhead
  1     & Identifier                   & 1
        & Unique string that identifies the instrument instance & \\
  \hline
  1.1   & identifierType               & 1
        & Type of the identifier & \\
  \hline
  2     & SchemaVersion                & 1
        & Version number of the PIDINST schema used in this record
        & Fixed value: \schemaversion \\
  \hline
  3     & LandingPage                  & 1
        & A landing page that the identifier resolves to & URL \\
  \hline
  4     & Name                         & 1
        & Name by which the instrument instance is known & Free text \\
  \hline
  5     & Owner                        & 1-n
        & Institution(s) responsible for the management of the
          instrument.  This may include the legal owner, the operator,
          or an institute providing access to the instrument.
        & \\
  \hline
  5.1   & ownerName                    & 1
        & Full name of the owner & Free text \\
  \hline
  5.2   & ownerContact                 & 0-1
        & Contact address of the owner & Electronic mail address \\
  \hline
  5.3   & ownerIdentifier              & 0-1
        & Identifier used to identify the owner
        & Free text, should be a globally unique identifier \\
  \hline
  5.3.1 & ownerIdentifierType          & 1
        & Type of the identifier & Free text \\
  \hline
  6     & Manufacturer                 & 1-n
        & The instrument's manufacturer(s) or developer.  This may
          also be the owner for custom build instruments.
        & \\
  \hline
  6.1   & manufacturerName             & 1
        & Full name of the manufacturer & Free text \\
  \hline
  6.2   & manufacturerIdentifier       & 0-1
        & Identifier used to identify the manufacturer
        & Free text, should be a globally unique identifier \\
  \hline
  6.2.1 & manufacturerIdentifierType   & 1
        & Type of the identifier & Free text \\
  \hline
  7     & Model                        & 0-1
        & Name of the model or type of device as attributed by the
          manufacturer
        & \\
  \hline
  7.1   & modelName                    & 1
        & Full name of the model & Free text \\
  \hline
  7.2   & modelIdentifier              & 0-1
        & Identifier used to identify the model
        & Free text, should be a globally unique identifier \\
  \hline
  7.2.1 & modelIdentifierType          & 1
        & Type of the identifier & Free text \\
  \hline
  8     & Description                  & 0-1
        & Technical description of the device and its capabilities
        & Free text \\
  \hline
  9     & InstrumentType               & 0-n
        & Classification of the type of the instrument & \\
  \hline
  9.1   & instrumentTypeName           & 1
        & Full name of the instrument type & Free text \\
  \hline
  9.2   & instrumentTypeIdentifier     & 0-1
        & Identifier used to identify the type of the instrument
        & Free text, should be a globally unique identifier \\
  \hline
  9.2.1 & instrumentTypeIdentifierType & 1
        & Type of the identifier & Free text \\
  \hline
  10    & MeasuredVariable             & 0-n
        & The variable(s) that this instrument measures or observes
        & Free text \\
  \hline
  11    & Date                         & 0-n
        & Dates relevant to the instrument & ISO 8601 \\
  \hline
  11.1  & dateType                     & 1
        & The type of the date
        & \begin{minipage}[t]{\valcolw}
            Controlled list of values:
            \begin{itemize}[nosep,leftmargin=3.5ex]
            \item Commissioned
            \item DeCommissioned
            \end{itemize}
            \vspace{1ex}
          \end{minipage} \\
  \hline
  12    & RelatedIdentifier            & 0-n
        & Identifiers of related resources & Free text, must be globally unique identifiers \\
  \hline
  12.1  & relatedIdentifierType        & 1
        & Type of the identifier
        & \begin{minipage}[t]{\valcolw}
            Controlled list of values:
            \begin{itemize}[nosep,leftmargin=3.5ex]
            \item ARK
            \item arXiv
            \item bibcode
            \item DOI
            \item EAN13
            \item EISSN
            \item Handle
            \item IGSN
            \item ISBN
            \item  ISSN
            \item ISTC
            \item LISSN
            \item PMID
            \item PURL
            \item RAiD
            \item RRID
            \item UPC
            \item URL
            \item URN
            \item w3id
            \end{itemize}
            \vspace{1ex}
          \end{minipage} \\
  \hline
  12.2  & relationType                 & 1
        & Description of the relationship
        & \begin{minipage}[t]{\valcolw}
            Controlled list of values:
            \begin{itemize}[nosep,leftmargin=3.5ex]
            \item IsDescribedBy
            \item IsNewVersionOf
            \item IsPreviousVersionOf
            \item HasComponent
            \item IsComponentOf
            \item References
            \item HasMetadata
            \item WasUsedIn
            \item IsIdenticalTo
            \item IsAttachedTo
            \end{itemize}
            \vspace{1ex}
          \end{minipage} \\
  \hline
  12.3  & relatedIdentifierName        & 0-1
        & A name for the related resource, may be used to give a hint on the content of that resource & Free text \\
  \hline
  13    & AlternateIdentifier          & 0-n
        & Identifiers other than the PIDINST pertaining to the same
          instrument instance.  This should be used if the instrument
          has a serial number.  Other possible uses include an owner's
          inventory number or an entry in some instrument data base.
        & Free text, should be unique identifiers \\
  \hline
  13.1  & alternateIdentifierType      & 1
        & Type of the identifier
        & \begin{minipage}[t]{\valcolw}
            Controlled list of values:
            \begin{itemize}[nosep,leftmargin=3.5ex]
            \item SerialNumber
            \item InventoryNumber
            \item Other
            \end{itemize}
            \vspace{1ex}
          \end{minipage} \\
  \hline
  13.2  & alternateIdentifierName      & 0-1
        & A supplementary name for the identifier type.  This is
          mostly useful if alternateIdentifierType is Other.
        & Free text \\
  \hline
\end{longtable}

\bibliographystyle{plain}
\bibliography{schema}

\end{document}
