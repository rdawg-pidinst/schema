\documentclass[titlepage=true,twoside=false,DIV=13]{scrartcl}
\usepackage[utf8]{inputenc}
\usepackage[T1]{fontenc}
\usepackage[colorlinks=true,urlcolor=black,linkcolor=black]{hyperref}
\usepackage{longtable}
\usepackage{enumitem}
\usepackage{pidinsttitle}

\input{.meta}

\begin{document}

\title{Metadata Schema for the \\
  Persistent Identification of Instruments}
\author{RDA Persistent Identification of Instruments WG}
\date{\schemadate}
\publishers{
  \begin{tabular}[t]{|l|l|}
    \hline
    Version & \schemaversion \\
    \hline
    Date    & \schemadate    \\
    \hline
  \end{tabular}
}

\renewcommand{\arraystretch}{1.3}

\maketitle

\pagenumbering{roman}
\setcounter{page}{2}

\section*{Contributors}

\begin{itemize}\emergencystretch 3em
\item \emph{Rolf Krahl}
  ({\small
  \href{mailto:rolf.krahl@helmholtz-berlin.de}{rolf.krahl@helmholtz-berlin.de},
  \url{https://orcid.org/0000-0002-1266-3819}}),
  Helmholtz-Zentrum Berlin für Materialien und Energie,
  Albert-Einstein-Str.\ 15, 12489 Berlin, Germany
\item \emph{Louise Darroch}
  ({\small
  \href{mailto:louise.darroch@bodc.ac.uk}{louise.darroch@bodc.ac.uk},
  \url{https://orcid.org/0000-0003-4163-9575}}),
  British Oceanographic Data Centre, National Oceanography Centre,
  Liverpool, L3 5DA, United Kingdom
\item \emph{Robert Huber}
  ({\small
  \href{mailto:rhuber@uni-bremen.de}{rhuber@uni-bremen.de},
  \url{https://orcid.org/0000-0003-3000-0020}}),
  MARUM - Center for Marine Environmental Sciences, University of Bremen,
  Leobener Str.\ 8, 28359 Bremen, Germany
\item \emph{Anusuriya Devaraju}
  ({\small
  \href{mailto:a.devaraju@uq.edu.au}{a.devaraju@uq.edu.au},
  \url{https://orcid.org/0000-0003-0870-3192}}),
  Terrestrial Ecosystem Research Network (TERN), The University of
  Queensland, Long Pocket Precinct, Level 5 Foxtail Building \#1019,
  80 Meiers Road, Indooroopilly QLD 4068, Australia
\item \emph{Jens Klump}
  ({\small
  \href{mailto:jens.klump@csiro.au}{jens.klump@csiro.au},
  \url{https://orcid.org/0000-0001-5911-6022}}),
  CSIRO Mineral Resources, 26 Dick Perry Avenue, Kensington WA 6151,
  Australia
\item \emph{Ted Habermann}
  ({\small
  \href{mailto:ted@metadatagamechangers.com}{ted@metadatagamechangers.com},
  \url{http://orcid.org/0000-0003-3585-6733}}),
  Metadata Game Changers, Boulder, Colorado, USA 80304
\item \emph{Markus Stocker}
  ({\small
  \href{mailto:markus.stocker@tib.eu}{markus.stocker@tib.eu},
  \url{https://orcid.org/0000-0001-5492-3212}}),
  TIB -- Leibniz Information Centre for Science and Technology,
  Welfengarten 1 B, 30167 Hannover, Germany and
  Leibniz University Hannover, Welfengarten 1, 30167 Hannover, Germany
\item \emph{The Research Data Alliance Persistent Identification of Instruments
  Working Group members}
  ({\small \url{https://www.rd-alliance.org/node/57186/members}})
\end{itemize}

\tableofcontents
\cleardoublepage
\pagenumbering{arabic}

\section{Introduction}
\label{intro}

Instruments, such as sensors used in environmental science, DNA
sequencers used in life sciences or laboratory engines used for
medical domains, are widespread in most fields of applied sciences.
The ability to link a physical instrument to the data that it
generated and contextual metadata such as when, where and how it was
operated, is critical for the accurate interpretation of that data.

The Persistent Identification of Instruments Working Group (PIDINST)
is a working group of the Research Data Alliance (RDA).  It aims to
establish a cross-discipline, practical solution for the globally
unique and persistent identification of instruments operated in the
sciences that can act as an anchor for associated metadata.  The group
considers instrument instances, e.g.\ the individual physical objects,
as opposed to instrument types or models.  See \cite{pidinst2020} for
an overview of the work developed by the RDA PIDINST WG.

The solution uses a resolvable persistent identifier (PID) published
with descriptive metadata that provides enough information to identify
the instrument.  Instrument PIDs also provide more context to the
instrument itself by linking other instruments and platforms,
descriptive information and other external metadata about the
instrument, and manufacturer and owner of the instrument.  This
information can then be automatically aggregated following the
so-called PID Graph \cite{pidgraph2021}.

This document defines the metadata schema that PIDINST developed for
the description of an instrument instance.  Institutions or
organizations operating scientific instruments may use the schema when
registering the metadata describing their instrument along with the
persistent identifier through a relevant PID infrastructure.  The
schema is publicly available for reuse and maintained in the code
repository of the working group \cite{pidinst:github}.

\section{Metadata Schema}
\label{schema}

The PIDINST metadata schema provides core properties to represent
instrument descriptions.  It contains essential information to
identify the individual instrument instance and to link the instance
with other sources of information and related entities.

The Working Group strives for a metadata schema that is applicable to
various kinds of instruments being used in all scientific domains.
For this reason, we kept the schema generic.  Technical details such
as specifications, data sheets, or detailed technique descriptions
were not included as these may only apply to a subclass of instruments
relevant to a particular scientific subdomain.  The core metadata of
an instrument may be extended with more detailed, community-specific
descriptions by linking the PIDINST record with supplementary metadata
using the properties RelatedIdentifier with relationType IsDescribedBy
or HasMetadata provided in the schema.

\subsection{Overview}

Table \ref{tab:schema:propoverview} provides an overview of the
properties in the PIDINST schema.  The last column indicates whether
the property is:
\begin{itemize}
\item Mandatory (M): the property must be provided for the metadata
  record to be valid,
\item Recommended (R): the property is optional, but it is recommended
  to be added if applicable, or
\item Optional (O): the property may be added to provide a richer
  description.
\end{itemize}
In the present version of the schema, all properties are either
mandatory or recommended.  It is envisioned that additional optional
properties (in addition to mandatory and recommended properties) will
be added in the future.

\begin{longtable}{|l|l|l|}
  \hline
  \emph{ID} & \emph{Property} & \emph{Obligation} \\
  \hline \endhead
  \hline \endfoot\endlastfoot
  1     & Identifier          & M \\
  2     & SchemaVersion       & M \\
  3     & LandingPage         & M \\
  4     & Name                & M \\
  5     & Owner               & M \\
  6     & Manufacturer        & M \\
  7     & Model               & R \\
  8     & Description         & R \\
  9     & InstrumentType      & R \\
  10    & MeasuredVariable    & R \\
  11    & Date                & R \\
  12    & RelatedIdentifier   & R \\
  13    & AlternateIdentifier & R \\
  \hline
  \caption{Overview of PIDINST properties}
  \label{tab:schema:propoverview}
\end{longtable}

If a value for any of the mandatory properties is not available, one
of the standard values for unknown information may be used instead,
see Appendix \ref{appendix:unknown}.

\subsection{Definition of the PIDINST properties}

Table \ref{tab:schema:expandprops} provides a detailed description for
the properties in the PIDINST metadata schema.  Simple properties,
such as Name and Description, take just one value.  Complex properties
are composed of multiple sub-properties.  For instance, Owner is
composed of ownerName, ownerContact, and ownerIdentifier, the latter
in turn having another sub-property ownerIdentifierType.  The
structure of properties and sub-properties is indicated in the
numbering in the ID column, e.g.\ the property 5.3 (ownerIdentifier)
has a sub-property 5.3.1 (ownerIdentifierType).

The number of occurrences for any property in one PIDINST metadata
record is indicated in the column labeled ``Occ''.  The meaning is:
\begin{itemize}
\item 1: a mandatory property that must appear once,
\item 0--1: an optional property that may appear at most once,
\item 1--n: a mandatory property with multiple values, e.g.\ the
  property may appear one or more times in a single record,
\item 0--n: a multivalued optional property that may appear zero
  or more times in a record.
\end{itemize}
In the case of a sub-property, the occurrence is to be understood
within each of the parent entries.  For example, a PIDINST record may
have zero or more RelatedIdentifier (occ.\ 0--n).  Each of them must
have one relatedIdentifierType (occ.\ 1) and one relationType
(occ.\ 1) and may optionally have one relatedIdentifierName
(occ.\ 0--1).  Another example: a PIDINST record must have one or more
Manufacturer (occ.\ 1--n).  Each of them must have a manufacturerName
(occ.\ 1) and may have a manufacturerIdentifier (occ.\ 0--1).  If a
manufacturerIdentifier is included for a Manufacturer, it must also
have a manufacturerIdentifierType (occ.\ 1).

\newlength{\idcolw}\settowidth{\idcolw}{5.3.1}
\newlength{\propcolw}\settowidth{\propcolw}{instrumentTypeIdentifierType}
\newlength{\occcolw}\settowidth{\occcolw}{\emph{Occ}}
\newlength{\valcolw}\settowidth{\valcolw}{Controlled list of values:}
\newlength{\defcolw}
\setlength{\defcolw}{\textwidth}
\addtolength{\defcolw}{-\idcolw}
\addtolength{\defcolw}{-\propcolw}
\addtolength{\defcolw}{-\occcolw}
\addtolength{\defcolw}{-\valcolw}
\addtolength{\defcolw}{-10\tabcolsep}

\begin{longtable}{|l|l|l|p{\defcolw}|p{\valcolw}|}
  \hline
  \emph{ID} & \emph{Property}          & \emph{Occ}
        & \emph{Definition}
        & \emph{Allowed values, constraints, remarks} \\
  \hline \endhead
  1     & Identifier                   & 1
        & Unique string that identifies the instrument instance & \\
  \hline
  1.1   & identifierType               & 1
        & Type of the identifier
        & The type of the identifier depends on the provider being
          used to register it.  In the case of ePIC, the value would
          be ``Handle''. \\
  \hline
  2     & SchemaVersion                & 1
        & Version number of the PIDINST schema used in this record
        & Fixed value: ``\schemaversion'' \\
  \hline
  3     & LandingPage                  & 1
        & A landing page that the identifier resolves to & URL \\
  \hline
  4     & Name                         & 1
        & Name by which the instrument instance is known & Free text \\
  \hline
  5     & Owner                        & 1--n
        & Institution(s) responsible for the management of the
          instrument.  This may include the legal owner, the operator,
          or an institute providing access to the instrument.
        & \\
  \hline
  5.1   & ownerName                    & 1
        & Full name of the owner & Free text \\
  \hline
  5.2   & ownerContact                 & 0--1
        & Contact address of the owner & Electronic mail address \\
  \hline
  5.3   & ownerIdentifier              & 0--1
        & Identifier used to identify the owner
        & Free text, should be a globally unique identifier \\
  \hline
  5.3.1 & ownerIdentifierType          & 1
        & Type of the identifier
        & Free text, see note below \\
  \hline
  6     & Manufacturer                 & 1--n
        & The instrument's manufacturer(s) or developer.  This may
          also be the owner for custom-build instruments.
        & \\
  \hline
  6.1   & manufacturerName             & 1
        & Full name of the manufacturer & Free text \\
  \hline
  6.2   & manufacturerIdentifier       & 0--1
        & Identifier used to identify the manufacturer
        & Free text, should be a globally unique identifier \\
  \hline
  6.2.1 & manufacturerIdentifierType   & 1
        & Type of the identifier
        & Free text, see note below \\
  \hline
  7     & Model                        & 0--1
        & Name of the model or type of device as attributed by the
          manufacturer
        & \\
  \hline
  7.1   & modelName                    & 1
        & Full name of the model & Free text \\
  \hline
  7.2   & modelIdentifier              & 0--1
        & Identifier used to identify the model
        & Free text, should be a globally unique identifier \\
  \hline
  7.2.1 & modelIdentifierType          & 1
        & Type of the identifier
        & Free text, see note below \\
  \hline
  8     & Description                  & 0--1
        & Technical description of the device and its capabilities
        & Free text \\
  \hline
  9     & InstrumentType               & 0--n
        & Classification of the type of the instrument & \\
  \hline
  9.1   & instrumentTypeName           & 1
        & Full name of the instrument type & Free text, see note below \\
  \hline
  9.2   & instrumentTypeIdentifier     & 0--1
        & Identifier used to identify the type of the instrument
        & Free text, should be a globally unique identifier \\
  \hline
  9.2.1 & instrumentTypeIdentifierType & 1
        & Type of the identifier
        & Free text, see note below \\
  \hline
  10    & MeasuredVariable             & 0--n
        & The variable(s) that this instrument measures or observes
        & Free text \\
  \hline
  11    & Date                         & 0--n
        & Dates relevant to the instrument & ISO 8601 \\
  \hline
  11.1  & dateType                     & 1
        & The type of the date
        & \begin{minipage}[t]{\valcolw}
            Controlled list of values:
            \begin{itemize}[nosep,leftmargin=3.5ex]
            \item Commissioned
            \item DeCommissioned
            \end{itemize}
            \vspace{1ex}
          \end{minipage} \\
  \hline
  12    & RelatedIdentifier            & 0--n
        & Identifiers of related resources
        & Free text, must be globally unique identifiers \\
  \hline
  12.1  & relatedIdentifierType        & 1
        & Type of the identifier
        & \begin{minipage}[t]{\valcolw}
            Controlled list of values:
            \begin{itemize}[nosep,leftmargin=3.5ex]
            \item ARK
            \item arXiv
            \item bibcode
            \item DOI
            \item EAN13
            \item EISSN
            \item Handle
            \item IGSN
            \item ISBN
            \item ISSN
            \item ISTC
            \item LISSN
            \item PMID
            \item PURL
            \item RAiD
            \item RRID
            \item UPC
            \item URL
            \item URN
            \item w3id
            \end{itemize}
            \vspace{1ex}
          \end{minipage} \\
  \hline
  12.2  & relationType                 & 1
        & Description of the relationship
        & \begin{minipage}[t]{\valcolw}
            Controlled list of values:
            \begin{itemize}[nosep,leftmargin=3.5ex]
            \item IsDescribedBy
            \item IsNewVersionOf
            \item IsPreviousVersionOf
            \item HasComponent
            \item IsComponentOf
            \item References
            \item HasMetadata
            \item WasUsedIn
            \item IsIdenticalTo
            \item IsAttachedTo
            \end{itemize}
            \vspace{1ex}
          \end{minipage} \\
  \hline
  12.3  & relatedIdentifierName        & 0--1
        & A name for the related resource, may be used to give a hint
          on the content of that resource
        & Free text \\
  \hline
  13    & AlternateIdentifier          & 0--n
        & Identifiers other than the PIDINST pertaining to the same
          instrument instance.  This should be used if the instrument
          has a serial number.  Other possible uses include an owner's
          inventory number or an entry in some instrument data base.
        & Free text, should be unique identifiers \\
  \hline
  13.1  & alternateIdentifierType      & 1
        & Type of the identifier
        & \begin{minipage}[t]{\valcolw}
            Controlled list of values:
            \begin{itemize}[nosep,leftmargin=3.5ex]
            \item SerialNumber
            \item InventoryNumber
            \item Other
            \end{itemize}
            \vspace{1ex}
          \end{minipage} \\
  \hline
  13.2  & alternateIdentifierName      & 0--1
        & A supplementary name for the identifier type.  This is
          mostly useful if alternateIdentifierType is Other.
        & Free text \\
  \hline
  \caption{Expanded PIDINST properties}
  \label{tab:schema:expandprops}
\end{longtable}

There is not one single comprehensive classification for instrument
types that would be applicable to all kinds of instruments being used
in all scientific domains.  That is why we can't prescribe a
controlled list of values for instrumentTypeName, but allow free text.
There are however controlled vocabularies or ontologies to classify
instrument types for some scientific domains.  If the instrument is
used in a domain that has a well established instrument type
classification, it is recommended to use these terms in
instrumentTypeName.  If furthermore the classification defines
identifiers for the terms, it is recommended to link them using
instrumentTypeIdentifier.

It is important that links to related resources can be resolved in a
reliable way.  Therefore, well established persistent identifier
schemes should preferably be used to make these links and that is why
we prescribe a controlled list of values for relatedIdentifierType.
For ownerIdentifierType, manufacturerIdentifierType,
modelIdentifierType, and instrumentTypeIdentifierType we anticipate
that there might be other, less well established identifier types that
may need to be used in some cases.  Hence we allow free text.
Nevertheless, the values defined for relatedIdentifierType should also
be used here, whenever it is possible.

As formalized controlled vocabularies evolve and become common
standard across domains, future versions of this schema may adopt them
and change properties that are free text in the current version to use
a controlled list of values instead.

\subsection{Definition of terms in controlled lists of values}

For some properties, the allowed values are constrained to a
controlled list.  In the following, we define the semantics of the
terms and provide guidance on when to use them.

\newlength{\valnamecolw}\settowidth{\valnamecolw}{IsPreviousVersionOf}
\newlength{\valdefcolw}
\setlength{\valdefcolw}{\textwidth}
\addtolength{\valdefcolw}{-\valnamecolw}
\addtolength{\valdefcolw}{-4\tabcolsep}

\subsubsection{dateType}

The Date property allows references to relevant events in the lifetime
of the instrument.  The dateType sub-property indicates the nature of
that event.  Table \ref{tab:schema:values:dateType} lists the allowed
values for dateType.

\begin{longtable}{|l|l|}
  \hline
  \emph{Value} & \emph{Definition} \\
  \hline \endhead
  \hline \endfoot\endlastfoot
  Commissioned   & The date when the instrument started to be in operation \\
  DeCommissioned & The date when the instrument ceased to be in operation \\
  \hline
  \caption{Values for dateType}
  \label{tab:schema:values:dateType}
\end{longtable}

\subsubsection{relatedIdentifierType}

The RelatedIdentifier property should be used to link other resources
of any kind related to the instrument, referring to a persistent
identifier of this resource.  The relatedIdentifierType sub-property
indicates the type of persistent identifier being used in the link.
The allowed values for relatedIdentifierType are listed in Table
\ref{tab:schema:values:relatedIdentifierType}.  Most of these values
have been adopted from the DataCite Metadata Schema
\cite{datacite:schema}.

\begin{longtable}{|l|p{\valdefcolw}|}
  \hline
  \emph{Value} & \emph{Definition} \\
  \hline \endhead
  \hline \endfoot\endlastfoot
  ARK     &
  Archival Resource Key: a URI designed to support long-term access to
  information objects.
  \\ \hline
  arXiv   &
  arXiv identifier: arXiv.org is a repository of preprints of
  scientific papers in the fields of mathematics, physics, astronomy,
  computer science, quantitative biology, statistics, and quantitative
  finance.
  \\ \hline
  bibcode &
  Astrophysics Data System bibliographic codes: a standardized
  19-character identifier.
  \\ \hline
  DOI     &
  Digital Object Identifier: a character string used to uniquely
  identify an object.  A DOI name is divided into two parts, a prefix
  and a suffix, separated by a slash.
  \\ \hline
  EAN13   &
  European Article Number, now renamed International Article Number,
  but retaining the original acronym, is a 13-digit barcoding standard
  that is a superset of the original 12-digit Universal Product Code
  (UPC) system.
  \\ \hline
  EISSN   &
  Electronic International Standard Serial Number: ISSN used to
  identify periodicals in electronic form (eISSN or e-ISSN).
  \\ \hline
  Handle  &
  This refers specifically to an ID in the Handle system operated by
  the Corporation for National Research Initiatives (CNRI).
  \\ \hline
  IGSN    &
  A persistent unique identifier for physical samples and specimens.
  See \url{https://www.igsn.org/}
  \\ \hline
  ISBN    &
  International Standard Book Number: a unique numeric book identifier.
  \\ \hline
  ISSN    &
  International Standard Serial Number: a unique 8-digit number used
  to identify a print or electronic periodical publication.
  \\ \hline
  ISTC    &
  International Standard Text Code: a unique ``number'' assigned to a
  textual work.  An ISTC consists of 16 numbers and/or letters.
  \\ \hline
  LISSN   &
  The linking ISSN or ISSN-L enables collocation or linking among
  different media versions of a continuing resource.
  \\ \hline
  PMID    &
  PubMed identifier: a unique number assigned to each PubMed record.
  \\ \hline
  PURL    &
  Persistent Uniform Resource Locator.
  \\ \hline
  RAiD    &
  Research Activity Identifier, see \url{https://www.raid.org.au/}
  \\ \hline
  RRID    &
  Research Resource Identifiers, see \url{https://www.rrids.org/}
  \\ \hline
  UPC     &
  Universal Product Code is a barcode symbology used for tracking
  trade items in stores.
  \\ \hline
  URL     &
  Uniform Resource Locator, also known as web address.
  \\ \hline
  URN     &
  Uniform Resource Name: a unique and persistent identifier of an
  electronic document.
  \\ \hline
  w3id    &
  Permanent identifier for Web applications.  Mostly used to publish
  vocabularies and ontologies.
  \\ \hline
  \caption{Values for relatedIdentifierType}
  \label{tab:schema:values:relatedIdentifierType}
\end{longtable}

\subsubsection{relationType}

The RelatedIdentifier property should be used to link other resources
of any kind related to the instrument, referring to a persistent
identifier of this resource.  The relationType sub-property indicates
the nature of that relation.  Table
\ref{tab:schema:values:relationType} lists the allowed values for
relationType.

\begin{longtable}{|l|p{\valdefcolw}|}
  \hline
  \emph{Value} & \emph{Definition} \\
  \hline \endhead
  \hline \endfoot\endlastfoot
  IsDescribedBy       &
  The linked resource is a document describing the instrument.
  \\ \hline
  IsNewVersionOf      &
  If an instrument is substantially modified, a new PID may be
  attributed to the new version.  In that case the old and the new PID
  should be linked to each other.  IsNewVersionOf should be used in
  the new PID record to link the old instrument before the
  modification.
  \\ \hline
  IsPreviousVersionOf &
  If an instrument is substantially modified, a new PID may be
  attributed to the new version.  In that case the old and the new PID
  should be linked to each other.  IsPreviousVersionOf should be used
  in the old PID record to link the new instrument after the
  modification.
  \\ \hline
  HasComponent        &
  In the case of a complex instrument, having multiple components that
  may be considered as instruments in their own right, with their own
  PIDs, these PIDs should be linked.  HasComponent should be used in
  the PID record of the compound instrument to link the components.
  \\ \hline
  IsComponentOf       &
  In the case of a complex instrument, having multiple components that
  may be considered as instruments in their own right, with their own
  PIDs, these PIDs should be linked.  IsComponentOf should be used in
  the PID records of the components to link the compound instrument.
  \\ \hline
  References          &
  This may be used in the generic case, if no other more specific
  relation type applies.
  \\ \hline
  HasMetadata         &
  If there is additional metadata describing the instrument, possibly
  using a community specific metadata standard, that metadata record
  may be linked using HasMetadata.
  \\ \hline
  WasUsedIn           &
  If the instrument has been deployed in some research activity, such
  as a cruise of a research vessel, WasUsedIn may be used to link that
  activity.
  \\ \hline
  IsIdenticalTo       &
  If multiple PIDs have been attributed to a single instrument (which
  should preferably be avoided in the first place), these PID records
  should be linked to each other using IsIdenticalTo.
  \\ \hline
  IsAttachedTo        &
  If the instrument is permanently attached to another instrument, the
  PID records for both instruments should link to each other using
  IsAttachedTo.
  \\ \hline
  \caption{Values for relationType}
  \label{tab:schema:values:relationType}
\end{longtable}

\subsubsection{alternateIdentifierType}

If the instrument is also registered elsewhere using any other kind of
identifier, these identifiers may be stored in the AlternateIdentifier
property.  The alternateIdentifierType sub-property indicates the
nature of that identifier.  The allowed values are listed in Table
\ref{tab:schema:values:alternateIdentifierType}.

\begin{longtable}{|l|l|}
  \hline
  \emph{Value} & \emph{Definition} \\
  \hline \endhead
  \hline \endfoot\endlastfoot
  SerialNumber    & A serial number attributed by the manufacturer \\
  InventoryNumber & An inventory number used by the owner \\
  Other           & Any other kind of identifier \\
  \hline
  \caption{Values for alternateIdentifierType}
  \label{tab:schema:values:alternateIdentifierType}
\end{longtable}

\appendix

\section{Standard values for unknown information}
\label{appendix:unknown}

For the case that the value for a mandatory property is unknown or not
available, we adopt the standard values for unknown information from
the DataCite Metadata Schema \cite{datacite:schema}.  These values are
repeated here in Table \ref{tab:appendix:unknown}.

\begin{longtable}{|l|l|}
  \hline
  \emph{Code} & \emph{Definition} \\
  \hline \endhead
  \hline \endfoot\endlastfoot
  :unac & Temporarily inaccessible \\
  :unal & Unallowed, suppressed intentionally \\
  :unap & Not applicable, makes no sense \\
  :unas & Value unassigned (e.g., Untitled) \\
  :unav & Value unavailable, possibly unknown \\
  :unkn & Known to be unknown (e.g., Anonymous, Inconnue) \\
  :none & Never had a value, never will \\
  :null & Explicitly and meaningfully empty \\
  :tba  & To be assigned or announced later \\
  :etal & Too numerous to list (et alia) \\
  \hline
  \caption{Standard values for unknown information}
  \label{tab:appendix:unknown}
\end{longtable}

%\bibliographystyle{plain}
%\bibliography{schema}
\begin{thebibliography}{1}

\bibitem{pidinst2020}
Markus Stocker et~al.
\newblock Persistent Identification of Instruments.
\newblock \emph{Data Science Journal}, 19(1), 18, 2020.
\newblock \url{https://doi.org/10.5334/dsj-2020-018}.

\bibitem{pidgraph2021}
Helena Cousijn et~al.
\newblock Connected Research: The Potential of the PID Graph.
\newblock \emph{Patterns}, 2(1), 100180, 2021.
\newblock \url{https://doi.org/10.1016/j.patter.2020.100180}.

\bibitem{pidinst:github}
RDA Persistent~Identification of~Instruments~WG.
\newblock Medadata Schema for the Persistent Identification of Scientific
  Instruments.
\newblock \url{https://github.com/rdawg-pidinst/schema}.
\newblock Code repository.

\bibitem{datacite:schema}
DataCite Metadata~Working Group.
\newblock DataCite Metadata Schema Documentation for the Publication and
  Citation of Research Data and Other Research Outputs v4.4, March 2021.
\newblock \url{https://doi.org/10.14454/3W3Z-SA82}.

\end{thebibliography}

\end{document}
